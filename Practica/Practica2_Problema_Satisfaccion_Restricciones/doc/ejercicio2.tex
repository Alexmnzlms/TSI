\section{Ejercicio 2}
\subsection{Representación}
Este problema se representa con un único array de 10 valores, en el que en cada posición de almacena un dígito del número. De esta manera la posición i almacena el número de i's que hay en el número.

\subsection{Restricciones}
La única restricción de este problema es simple: El número presente en la posición i debe ser igual al número de veces que aparece i en el array.

\subsection{Solución}
Vemos como MiniZinc encuentra la solución propuesta como ejemplo: 6210001000, 6 ceros, 2 unos, 1 dos y 1 seis.

\includegraphics[scale=0.5]{ej2.png}
