\section{Ejercicio 1}
\subsection{Representación}
La representación de este problema se implementa en base a dos array de enteros. El array \emph{letras} almacena los valores asignados a cada letra y el array \emph{acarreo} almacena el acarreo de las 5 sumas de letras que debemos realizar.

\subsection{Restricciones}
Las sumas individuales que se realizan en este puzzle son:
\begin{itemize}
   \item $ E + E + E = E$
   \item $ T + T + N + acarreo = T$
   \item $ S + S + I + acarreo = F$
   \item $ E + E + E + acarreo = A$
   \item $ T + F + D + acarreo = R$
   \item $ K = acarreo $
\end{itemize}

Estas son las restricciones que debemos satisfacer. Para esto, establecemos que cada suma de 3 letras debe ser igual al resultado correspondiente una vez aplicado la operación modulo 10.
\\
Por ejemplo: $ (E + E + E)\% 10 = E $.
\\
Además calculamos el acarreo para la siguiente suma como el resultado de la operación anterior, pero esta vez dividido por 10.
\\
En el mismo ejemplo: $acarreo = \frac{E+E+E}{10}$
\\
En la última suma sabemos que $K$ será igual al último $acarreo$ calculado.

\subsection{Solución}
MiniZinc encuentra la siguiente asignación de valores:\\
\\
$TESTE + FESTE + DEINE = 30830 + 60830 + 50970 = 142630 = KRAFTE $\\
\\
\includegraphics[scale=0.5]{ej1.png}
