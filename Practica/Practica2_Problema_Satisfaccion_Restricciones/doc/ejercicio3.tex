\section{Ejercicio 3}
\subsection{Representación}
La representación del problema se implementa mediante un array con tantas posiciones como profesores debemos asignar.

\subsection{Restricciones}
Las restricciones se pueden traducir literalmente acotando que para determinada i esta se encuentre en un rango determinado.

\subsection{Solución}

Las asignación que aporta MiniZinc es la siguiente:

\begin{figure}[h]
   \centering
   \begin{subfigure}[b]{0.45\textwidth}
      \includegraphics[scale=0.5]{ej3.png}
   \end{subfigure}
   \hfill
   \begin{subfigure}[b]{0.45\textwidth}
      \begin{table}[H]
      \resizebox{0.8\textwidth}{!}{%
      \begin{tabular}{|l|l|}
      \hline
      09:00 a 10:00 & Prof-6 \\ \hline
      10:00 a 11:00 & Prof-4 \\ \hline
      11:00 a 12:00 & Prof-5 \\ \hline
      12:00 a 13:00 & Prof-2 \\ \hline
      13:00 a 14:00 & Prof-3 \\ \hline
      14:00 a 15:00 & Prof-1 \\ \hline
      \end{tabular}
      }
      \end{table}
   \end{subfigure}
\end{figure}
