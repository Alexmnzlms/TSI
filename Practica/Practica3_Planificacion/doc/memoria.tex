En esta memoria se detallarán las decisiones tomadas en la realización de los ejercicios propuestos de planificación con PDDL. En el ejercicio 1 se detallarán las características del dominio en su totalidad, mientras que en los siguientes solo se hará mención a los cambios realzados en el mismo, entendiendo que las partes no mencionadas permanecen inalteradas.

\section{Ejercicio 1}
\subsection{Tipos}
Se han definido dos tipos principales: \textbf{Posicinable} y \textbf{Localizaciones}. El tipo Localizaciones representa las posibles posiciones dentro de la matriz del mundo dentro de la que las unidades, edificios y recursos pueden encontrarse. El tipo Posicionable es un supertipo para los subtipos de \textbf{unidad}, \textbf{edificio}, \textbf{tipoUnidad}, \textbf{tipoRecurso}, \textbf{tipoEdificio}. El supertipo Posicionable representa objetos del dominio/problema que pueden ocupar una posición en el mundo. Los subtipos unidad y edificio representan a unidades y edificios concretos, mientras que los tipos tipoX representan el tipo de la entidad X. Notar que existe un tipo llamado tipoRecurso pero no uno llamado recurso, esto es debido a que las unidades y edificios pueden tener instancias concretas, mientras que para representar los recursos simplemente necesitamos asociar un tipo de recurso a una casilla concreta, sin necesidad de crear una instancia concreta del mismo.

\subsubsection{Constantes}
Se han definido las constantes \textbf{CentroDeMando} y \textbf{Barracones} del tipo tipoEdificio, \textbf{VCE} del tipo tipoUnidad y \textbf{Minerales} y \textbf{Gas} del tipo tipoRecurso.

\subsection{Predicados}
\begin{itemize}
   \item El predicado \textit{EN} establece la relación entre una entidad de tipo Posicionable y una entidad de tipo Localizaciones, es decir, establece la posición de una entidad concreta en una localización concreta.
   \item El predicado \textit{CAMINO} establece la relación entre dos entidades de tipo Localizaciones. Esto permite establecer las conexiones entre las distintas posiciones que componen el mundo representado.
   \item El predicado \textit{EXTRAE} establece la relación entre una entidad de tipo unidad y otra de tipoRecurso. Esta relación establece que una unidad concreta se encuentra extrayendo un tipo de recurso.
   \item El predicado \textit{NECESITA} establece una relación entre una entidad tipoEdificio y otra de tipo tipoRecurso. Establece el tipo de recurso que necesita un tipo de edificio para poder construirse.
   \item El predicado \textit{ESTIPOEDIFICIO} establece una relación entre una entidad de tipo edificio y otra de tipoEdificio, es decir, especifica el tipo de edificio que es una construccion especifica.
   \item El predicado \textit{ESTIPOUNIDAD} establece una relación entre una entidad de tipo unidad y otra de tipoUnidad, es decir, especifica el tipo de unidad que es una unidad especifica.
\end{itemize}

\subsection{Acciones}
En este nivel se implementan las 3 acciones básicas \textbf{Navegar}, \textbf{Asignar} y \textbf{Construir}
\begin{itemize}
   \item \textbf{Navegar}:
      \begin{itemize}
         \item Parametros:
         \item Precondiciones:
         \item Efecto:
      \end{itemize}
   \item \textbf{Asignar}:
   \item \textbf{Construir}:
\end{itemize}
\subsection{Problema y solución}

\section{Ejercicio 2}

\subsection{Tipos}
\subsubsection{Constantes}
\subsection{Predicados}
\subsection{Acciones}
\subsection{Problema y solución}

\section{Ejercicio 3}

\subsection{Tipos}
\subsubsection{Constantes}
\subsection{Predicados}
\subsection{Acciones}
\subsection{Problema y solución}

\section{Ejercicio 4}

\subsection{Tipos}
\subsubsection{Constantes}
\subsection{Predicados}
\subsection{Acciones}
\subsection{Problema y solución}

\section{Ejercicio 5}

\subsection{Tipos}
\subsubsection{Constantes}
\subsection{Predicados}
\subsection{Acciones}
\subsection{Problema y solución}

\section{Ejercicio 6}

\subsection{Tipos}
\subsubsection{Constantes}
\subsection{Predicados}
\subsection{Acciones}
\subsection{Problema y solución}
