En esta memoria se detallarán las decisiones tomadas en la realización de los ejercicios propuestos de planificación con PDDL. En el ejercicio 1 se detallarán las características del dominio en su totalidad, mientras que en los siguientes solo se hará mención a los cambios realzados en el mismo.

\section{Ejercicio 1}
\subsection{Tipos y constantes}
Se han definido dos tipos principales: \textbf{posicionable} y \textbf{localizaciones}. El tipo localizaciones representa las posibles posiciones dentro de la matriz del mundo dentro de la que las unidades, edificios y recursos pueden encontrarse. El tipo posicionable es un supertipo para los subtipos de \textbf{unidad}, \textbf{edificio}, \textbf{tipoUnidad}, \textbf{tipoRecurso}, \textbf{tipoEdificio}. El supertipo posicionable representa objetos del dominio/problema que pueden ocupar una posición en el mundo. Los subtipos unidad y edificio representan a unidades y edificios concretos, mientras que los tipos tipoX representan el tipo de la entidad X. Notar que existe un tipo llamado tipoRecurso pero no uno llamado recurso, esto es debido a que las unidades y edificios pueden tener instancias concretas, mientras que para representar los recursos simplemente necesitamos asociar un tipo de recurso a una casilla concreta, sin necesidad de crear una instancia concreta del mismo.
\\\\
Se han definido las constantes \textbf{CentroDeMando} y \textbf{Barracones} del tipo tipoEdificio, \textbf{VCE} del tipo tipoUnidad y \textbf{Minerales} y \textbf{Gas} del tipo tipoRecurso.

\subsection{Predicados}
\begin{itemize}
   \item El predicado \textit{EN} establece la relación entre una entidad de tipo posicionable y una entidad de tipo localizaciones, es decir, establece la posición de una entidad concreta en una localización concreta.
   \item El predicado \textit{CAMINO} establece la relación entre dos entidades de tipo localizaciones. Esto permite establecer las conexiones entre las distintas posiciones que componen el mundo representado.
   \item El predicado \textit{EXTRAE} establece la relación entre una entidad de tipo unidad y otra de tipoRecurso. Esta relación establece que una unidad concreta se encuentra extrayendo un tipo de recurso.
   \item El predicado \textit{NECESITA} establece una relación entre una entidad tipoEdificio y otra de tipo tipoRecurso. Establece el tipo de recurso que necesita un tipo de edificio para poder construirse.
   \item El predicado \textit{ESTIPOEDIFICIO} establece una relación entre una entidad de tipo edificio y otra de tipoEdificio, es decir, especifica el tipo de edificio que es una construcción especifica.
   \item El predicado \textit{ESTIPOUNIDAD} establece una relación entre una entidad de tipo unidad y otra de tipoUnidad, es decir, especifica el tipo de unidad que es una unidad especifica.
\end{itemize}

\subsection{Acciones}
En este nivel se implementan las 3 acciones básicas \textbf{Navegar}, \textbf{Asignar} y \textbf{Construir}
\begin{itemize}
   \item \textbf{Navegar}: Esta acción se encarga de mover una unidad de una localización a otra adyacente.
      \begin{itemize}
         \item Parámetros: una unidad concreta y dos localizaciones.
         \item Precondiciones: Debe existir un camino entre las dos localizaciones y la unidad debe estar en la primera localización.
         \item Efecto: La unidad esta en la segunda localización y ya no esta en la primera localización.
      \end{itemize}
   \item \textbf{Asignar}: Esta acción se encarga de asignar una unidad a un tipo de recurso para que lo extraiga en una localización concreta.
   \begin{itemize}
      \item Parámetros: Una unidad concreta, un tipo de recurso y una localización.
      \item Precondiciones: La unidad debe estar en la localización y también debe existir ese tipo de recurso en la misma.
      \item Efecto: La unidad extrae el tipo de recurso
   \end{itemize}
   \item \textbf{Construir}: Esta acción se encarga de construir un edificio en una localización.
   \begin{itemize}
      \item Parámetros: una unidad y edificio específicos, un tipo de edificio y una localización.
      \item Precondiciones: Para todo tipo de recurso la unidad que construye no esta extrayendo ninguno (la unidad esta libre). Existe una unidad distinta y un tipo de recurso tal que la unidad se encarga de extraer el recurso que se necesita para construir el tipo de edificio que se quiere construir. La unidad debe estar en la localización donde se quiere construir. El edificio debe ser del tipo de edificio deseado.
      \item Efecto: El edificio se encuentra construido en la localización.
   \end{itemize}
\end{itemize}
\subsection{Problema y solución}
El objetivo de este ejercicio es construir un barracón en una localización concreta.\\
Para la resolución del problema se establece un mundo de 5x5 de tamaño. Se definen 3 unidades del tipo VCE, un centro de tipo CentroDeMando y un barracón de tipo Barracones.\\
Se establecen los caminos existentes en la matriz y en que posiciones se encuentran las unidades, los tipos de recursos y los edificios. Finalmente se establece el tipo de recurso necesario para construir cada tipo de edificio.\\
Metric-FF encuentra un plan basado en mover un VCE a la posición donde se quiere construir el barracón y mover otro a una localización con minerales para extraerlos y poder construir el barracón.

\section{Ejercicio 2}
\subsection{Tipos y constantes}
Se ha añadido la constante \textbf{Extractor} para representar el nuevo tipo de edificio.
\subsection{Acciones}
\begin{itemize}
   \item \textbf{Asignar}:
   \begin{itemize}
      \item Parámetros: Una unidad concreta, un tipo de recurso y una localización.
      \item Precondiciones: La unidad debe estar en la localización y también debe existir ese tipo de recurso en la misma. Además deben cumplirse dos condiciones, o en la localización no hay un tipo de recurso Gas o existe un edificio en la localización es de tipo Extractor.
      \item Efecto: La unidad extrae el tipo de recurso.
   \end{itemize}
\end{itemize}
\subsection{Problema y solución}
El objetivo de este ejercicio es construir un barracón en una localización concreta.\\
Además de todo lo definido en el problema anterior, debemos definir el tipo de recurso necesario para construir un extractor.\\
Metric-FF encuentra un plan basado en mover un VCE a la posición donde se quiere construir el barracón y mover otro a una localización con minerales para extraerlos y poder construir el barracón.

\section{Ejercicio 3}
\subsection{Acciones}
\begin{itemize}
   \item \textbf{Construir}: Esta acción se encarga de construir un edificio en una localización.
   \begin{itemize}
      \item Parámetros: una unidad y edificio específicos, un tipo de edificio y una localización.
      \item Precondiciones: Para todo tipo de recurso la unidad que construye no esta extrayendo ninguno (la unidad esta libre). La unidad debe estar en la localización donde se quiere construir. El edificio debe ser del tipo de edificio deseado. Ahora se debe cumplir que para todo tipo de recuro si el tipo de edificio necesita el tipo de recurso para construirse, existe una unidad diferente a la que construye que se encarga de extraerlo. Esto lo haremos utilizando el operador imply que nos permite comprobar la segunda parte de una comprobación solo si la primera es cierta.
      \item Efecto: El edificio se encuentra construido en la localización.
   \end{itemize}
\end{itemize}
\subsection{Problema y solución}
El objetivo de este ejercicio es construir un barracón en una localización concreta.\\
Además de todo lo definido en el problema anterior, debemos añadir que ahora el centro de mando también necesita Minerales para construirse.\\
Metric-FF encuentra un plan basado en mover un VCE a la posición donde se quiere construir el barracón y mover otro a una localización con minerales para extraerlos y poder construir el barracón.


\section{Ejercicio 4}

\subsection{Tipos y constantes}
Se ha restructurado la jerarquia de tipos. Ahora se ha añadido el subtipo \textbf{construible} (del supertipo posicionable) del que tipoUnidad y tipoEdificio son subtipos.
\\\\
Se han definido las constantes \textbf{Marines} y \textbf{Segadores} del tipo tipoEdificio.
\subsection{Predicados}
\begin{itemize}
   \item El predicado \textit{NECESITA} establece ahora una relación entre una entidad construible y otra de tipo tipoRecurso. Establece el tipo de recurso que necesita un tipo de edificio o tipo de unidad para poder construirse.
   \item El predicado \textit{RECLUTA} establece una relación entre un tipo de edificio y un tipo de unidad. Se definen asi que tipo de edificio reclutan que tipo de unidades.
\end{itemize}
\subsection{Acciones}
\begin{itemize}
   \item \textbf{Construir}: Esta acción se encarga de construir un edificio en una localización.
   \begin{itemize}
      \item Parámetros: una unidad y edificio específicos, un tipo de edificio y una localización.
      \item Precondiciones: Para todo tipo de recurso la unidad que construye no esta extrayendo ninguno (la unidad esta libre). La unidad debe estar en la localización donde se quiere construir. El edificio debe ser del tipo de edificio deseado. Ahora se debe cumplir que para todo tipo de recuro si el tipo de edificio necesita el tipo de recurso para construirse, existe una unidad diferente a la que construye que se encarga de extraerlo y ademas esta unidad es un VCE. También la unidad que se encarga de construir debe ser de tipo VCE.
      \item Efecto: El edificio se encuentra construido en la localización.
   \end{itemize}
   \item \textbf{Reclutar}: Esta acción se encarga de reclutar una unidad.
   \begin{itemize}
      \item Parámetros: una unidad específica, un tipo de unidad, un tipo de edificio y una localización concreta.
      \item Precondiciones: El tipo de la unidad especifica coincide con el tipo de unidad que queremos reclutar. Para todo tipo de recurso, si el tipo de unidad que queremos reclutar necesita ese tipo de recurso, existe una unidad distinta de tipo VCE que se encarga de extraer ese recurso. Además existe un edificio concreto tal que es del tipo de edificio que recluta el tipo de unidad deseado y ademas esta en la localización.
      \item Efecto: la unidad reclutada está en la localización.
   \end{itemize}
\end{itemize}
\subsection{Problema y solución}
El objetivo de este ejercicio es tener un marine en una localización concreta y un segador y un marine en otra.\\
Además de todo lo definido en el problema anterior, ahora comenzamos solo con una unidad VCE y un centro de mando y debemos añadir que recursos son necesarios para reclutar a las unidades y definir dos unidades de tipo Marines y una de tipo Segadores. También hay que especificar que tipo de edificio recluta a cada unidad.\\
Metric-FF encuentra un plan basado en asignar el VCE a recoger minerales, reclutar otro VCE en el centro de mando, construir un extractor, construir un barracón y reclutar un marine en el, mover este marine a donde corresponde, asignar al VCE a extraer gas y reclutar otro marine y un segador.


\section{Ejercicio 5}

\subsection{Tipos y constantes}
Se ha añadido el tipo principal \textbf{investigacion} que representa las investigaciones existentes.
\\\\
También se ha añadido la constante \textbf{BahiaDeInvestigacion}.
\subsection{Predicados}
\begin{itemize}
   \item El predicado \textit{NECESITAINV} establece una relación entre una investigación y el tipo de recurso que es necesario para poder realizarla.
   \item El predicado \textit{INVESTIGADO} simplemente sirve para indicar si una investigación ha sido realizada.
   \item El predicado \textit{PERMITE} establece una relación entre una investigación y un tipo de unidad. Establece la necesidad de haber realizado una investigación previamente para poder reclutar una unidad.
\end{itemize}
\subsection{Acciones}
\begin{itemize}
   \item \textbf{Reclutar}: Esta acción se encarga de reclutar una unidad.
   \begin{itemize}
      \item Parámetros: una unidad específica, un tipo de unidad, un tipo de edificio y una localización concreta.
      \item Precondiciones: El tipo de la unidad especifica coincide con el tipo de unidad que queremos reclutar. Para todo tipo de recurso, si el tipo de unidad que queremos reclutar necesita ese tipo de recurso, existe una unidad distinta de tipo VCE que se encarga de extraer ese recurso. Además existe un edificio concreto tal que es del tipo de edificio que recluta el tipo de unidad deseado y ademas esta en la localización. Para toda investigación si esta permite reclutar al tipo de unidad que queremos reclutar, entonces se ha de haber realizado la investigación. Conseguiremos esto aplicando de nuevo el operador imply.
      \item Efecto: la unidad reclutada está en la localización.
   \end{itemize}
   \item \textbf{Investigar}:
   \begin{itemize}
      \item Parámetros: una entidad de tipo investigación
      \item Precondiciones: Para todo tipo de recurso, si la investigación necesita ese tipo de recurso, existe una unidad de tipo VCE que se encarga de extraer ese recurso. También deben existir una localización y un edificio tal que el edificio esta en la localización y es de tipo BahiaDeInvestigacion, es decir, para investigar debe existir una bahía de investigación.
      \item Efecto: Se ha realizado la investigación.
   \end{itemize}
\end{itemize}
\subsection{Problema y solución}
El objetivo de este ejercicio es tener un marine en una localización concreta y un segador y un marine en otra.\\
Además de todo lo definido en el problema anterior, debemos establecer que recursos son necesarios para realizar las investigaciones y que investigaciones permiten reclutar que tipo de unidades. También debemos definir y colocar una bahía de investigación\\
Metric-FF encuentra un plan basado en asignar el VCE a recoger minerales, reclutar otro VCE en el centro de mando, construir un extractor, construir un barracón y reclutar un marine en el, mover este marine a donde corresponde, construir una bahía de investigación e investigar la mejora segador impulsor ,asignar al VCE a extraer gas y reclutar otro marine y un segador.


\section{Ejercicio 6}
\subsection{Tipos y constantes}
Se ha añadido la constante \textbf{Deposito}.
\subsection{Funciones}
Para poder trabajar con datos numéricos, en PDDL es necesario definir funciones:
\begin{itemize}
   \item La función \textit{COSTE} contiene el coste según un tipo de recurso para una entidad de tipo construible.
   \item La función \textit{COSTEINV} contiene el coste según un tipo de recurso para una entidad de tipo investigación.
   \item La función \textit{INCREMENTALIMITE} contiene cuanto aumenta el limite máximo de recursos que se puede almacenar cuando se construye un edificio.
   \item La función \textit{LIMITE} contiene el limite de unidades de un tipo de recurso que se puede almacenar.
   \item La función \textit{ALMACENADO} contiene el numero de unidades almacenadas de un tipo de recurso.
\end{itemize}
\subsection{Acciones}
\begin{itemize}
   \item \textbf{Navegar}:
      \begin{itemize}
         \item Parámetros: una unidad concreta y dos localizaciones.
         \item Precondiciones: Debe existir un camino entre las dos localizaciones y la unidad debe estar en la primera localización. Ademas ahora debe ser cierto que para todo tipo de recurso, la unidad no esta extrayendo dicho tipo de recurso.
         \item Efecto: La unidad esta en la segunda localización y ya no esta en la primera localización.
      \end{itemize}
   \item \textbf{Asignar}:
      \begin{itemize}
         \item Parámetros: Una unidad concreta, un tipo de recurso y una localización.
         \item Precondiciones: La unidad debe estar en la localización y también debe existir ese tipo de recurso en la misma. Deben cumplirse dos condiciones, o en la localización no hay un tipo de recurso Gas o existe un edificio en la localización es de tipo Extractor. Ademas ahora debe ser cierto que para todo tipo de recurso, la unidad no esta extrayendo dicho tipo de recurso.
         \item Efecto: La unidad extrae el tipo de recurso.
      \end{itemize}
   \item \textbf{Desasignar}:
         \begin{itemize}
            \item Parámetros: Una unidad concreta, un tipo de recurso y una localización.
            \item Precondiciones: La unidad y el tipo de recurso se encuentran en la localización y la unidad esta extrayendo dicho tipo de recurso.
            \item Efecto: La unidad ya no extrae el tipo de recurso
         \end{itemize}
   \item \textbf{Construir}: Esta acción se encarga de construir un edificio en una localización.
      \begin{itemize}
         \item Parámetros: una unidad y edificio específicos, un tipo de edificio y una localización.
         \item Precondiciones: Para todo tipo de recurso la unidad que construye no esta extrayendo ninguno (la unidad esta libre). La unidad debe estar en la localización donde se quiere construir. El edificio debe ser del tipo de edificio deseado. Ahora se debe cumplir que para todo tipo de recuro si el tipo de edificio necesita el tipo de recurso para construirse, existe una unidad diferente a la que construye que se encarga de extraerlo y ademas esta unidad es un VCE y además el coste en ese tipo de recursos para construir el tipo de edificio es menor que el número de recursos almacenados de ese tipo y menor que el limite de recursos de ese tipo que podemos almacenar. De esta manera conseguiremos que el limite de recursos que podemos almacenar siempre sea mayor que el coste más alto, reduciendo el espacio de búsqueda. También la unidad que se encarga de construir debe ser de tipo VCE.
         \item Efecto: El edificio se encuentra construido en la localización. Cuando el tipo de edificio construido es Deposito, se suma al limite de cada tipoRecurso el INCREMENTALIMITE de Deposito. Se decrementa a ALMACENADO de cada tipo de recurso el coste en ese tipo de recurso de construir el edificio.
      \end{itemize}
   \item \textbf{Reclutar}: Esta acción se encarga de reclutar una unidad.
      \begin{itemize}
         \item Parámetros: una unidad específica, un tipo de unidad, un tipo de edificio y una localización concreta.
         \item Precondiciones: El tipo de la unidad especifica coincide con el tipo de unidad que queremos reclutar. Para todo tipo de recurso, si el tipo de unidad que queremos reclutar necesita ese tipo de recurso, existe una unidad distinta de tipo VCE que se encarga de extraer ese recurso y además el coste en ese tipo de recursos para reclutar el tipo de unidad es menor que el número de recursos almacenados de ese tipo y menor que el limite de recursos de ese tipo que podemos almacenar. Además existe un edificio concreto tal que es del tipo de edificio que recluta el tipo de unidad deseado y ademas esta en la localización. Para toda investigación si esta permite reclutar al tipo de unidad que queremos reclutar, entonces se ha de haber realizado la investigación.
         \item Efecto: la unidad reclutada está en la localización. Se decrementa a ALMACENADO de cada tipo de recurso el coste en ese tipo de recurso de reclutar la unidad.
      \end{itemize}
   \item \textbf{Investigar}:
      \begin{itemize}
         \item Parámetros: una entidad de tipo investigación
         \item Precondiciones: Para todo tipo de recurso, si la investigación necesita ese tipo de recurso, existe una unidad de tipo VCE que se encarga de extraer ese recurso y además el coste en ese tipo de recursos para realizar la investigación es menor que el número de recursos almacenados de ese tipo y menor que el limite de recursos de ese tipo que podemos almacenar. También deben existir una localización y un edificio tal que el edificio esta en la localización y es de tipo BahiaDeInvestigacion, es decir, para investigar debe existir una bahía de investigación.
         \item Efecto: Se ha realizado la investigación. Se decrementa a ALMACENADO de cada tipo de recurso el coste en ese tipo de recurso de realizar la investigación.
      \end{itemize}
   \item \textbf{Recolectar}:
         \begin{itemize}
            \item Parámetros: una unidad concreta, un tipo de recurso y una localización
            \item Precondiciones: La unidad extrae el tipo de recurso y la unidad esta en la localización. El limite de recursos almacenados de ese tipo debe ser mayor que los recursos almacenados de ese tipo.
            \item Efecto: Se incrementa ALMACENADO de ese tipo de recurso en 25 unidades. Se ha decidido aumentar el incremento de 10 a 25, porque con 10 el tiempo de ejecución de Metric-FF es inabarcable.
         \end{itemize}
\end{itemize}
\subsection{Problema y solución}
El objetivo de este ejercicio es tener un marine en una localización concreta y un segador y un marine en otra.\\
Además de todo lo definido en el problema anterior, debemos definir los valores de las funciones numéricas definidas previamente.\\
Metric-FF encuentra un plan que resuelve correctamente el problema, aunque debido a las restricciones numericas, el numero de pasos y de tiempo necesario para calcular el plan ha aumentado.
